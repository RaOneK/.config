\documentclass{article}
\LARGE
% Language setting
% Replace `english' with e.g. `spanish' to change the document language
\usepackage[english]{babel}

% Set page size and margins
% Replace `letterpaper' with `a4paper' for UK/EU standard size
\usepackage[letterpaper,top=2cm,bottom=2cm,left=3cm,right=3cm,marginparwidth=1.75cm]{geometry}


\usepackage{tikz}
\usetikzlibrary{shapes.geometric, arrows}
\tikzstyle{startstop} = [rectangle, rounded corners, minimum width=3cm, minimum height=1cm,text centered, draw=black, fill=red!30]
\tikzstyle{io} = [trapezium, trapezium left angle=70, trapezium right angle=110, minimum width=3cm, minimum height=1cm, text centered, draw=black, fill=blue!30]
\tikzstyle{process} = [rectangle, minimum width=3cm, minimum height=1cm, text centered, draw=black, fill=orange!30]
\tikzstyle{decision} = [diamond, minimum width=3cm, minimum height=1cm, text centered, draw=black, fill=green!30]
\tikzstyle{arrow} = [thick,->,>=stealth]




\pgfdeclarelayer{bg}
\pgfsetlayers{bg,main}

% Useful packages
\usepackage{amsmath}
\usepackage{amsfonts}
\usepackage{graphicx}
\usepackage[colorlinks=true, allcolors=cyan]{hyperref}
\numberwithin{equation}{section}
\usepackage{graphicx,wrapfig,lipsum,subfigure,sidecap,epsfig}
\usepackage{caption}
\usepackage{cancel}
%\usepackage{graphicx,subfigure,sidecap,epsfig} % Rouslan's subfig package
\usepackage{soul}
%\usepackage[colorlinks=true,linkcolor=red]{hyperref}%
\usepackage{mathtools}
\usepackage{eqparbox}
\usepackage{float} % \figure{}[H] IN PLACE VIEW
\usepackage[capitalize]{cleveref} % smart references in one bracket
\usepackage{hyperref}
\usepackage{amssymb} % rightleft arrows
\usepackage{cancel} % \cancelto{<value>}{expression} diagonally

\usepackage{dcolumn}
\newcolumntype{d}{D{i}{i}{0}}

%
% CREF rules
% Equation(s)
\crefformat{equation}{#2Eq. (#1)#3}
\crefrangeformat{equation}{#3Eqs. (#1)#4 to #5(#2)#6}
\crefmultiformat{equation}{#2Eqs. (#1)#3}{ and #2(#1)#3}{, #2(#1)#3}{ and #2(#1)#3}
\crefrangemultiformat{equation}{#3Eqs. ((#1))#4 to #5((#2))#6}{ and #3(#1)#4 to #5(#2)#6}{, #3(#1)#4 to #5(#2)#6}{ and #3(#1)#4 to #5(#2)#6}
% Plural eqn
\crefformat{pluralequation}{#2Eqs.~(#1)#3}
% System
\crefformat{system}{#2Sys.~(#1)#3}
\crefrangeformat{system}{#3Sys. (#1)#4 to #5(#2)#6}
\crefmultiformat{system}{#2Sys. (#1)#3}{ and #2(#1)#3}{, #2(#1)#3}{ and #2(#1)#3}
\crefrangemultiformat{system}{#3Sys. ((#1))#4 to #5((#2))#6}{ and #3(#1)#4 to #5(#2)#6}{, #3(#1)#4 to #5(#2)#6}{ and #3(#1)#4 to #5(#2)#6}
% Boundary conditions
\crefformat{bc}{#2BC (#1)#3}
\crefrangeformat{bc}{#3BCs (#1)#4 to #5(#2)#6}
\crefmultiformat{bc}{#2BCs (#1)#3}{ and #2(#1)#3}{, #2(#1)#3}{ and #2(#1)#3}
\crefrangemultiformat{bc}{#3BCs ((#1))#4 to #5((#2))#6}{ and #3(#1)#4 to #5(#2)#6}{, #3(#1)#4 to #5(#2)#6}{ and #3(#1)#4 to #5(#2)#6}
% Steps
\crefformat{step}{#2Step (#1)#3}
\crefrangeformat{step}{#3Steps (#1)#4 to #5(#2)#6}
\crefmultiformat{step}{#2Steps (#1)#3}{ and #2(#1)#3}{, #2(#1)#3}{ and #2(#1)#3}
\crefrangemultiformat{step}{#3Steps ((#1))#4 to #5((#2))#6}{ and #3(#1)#4 to #5(#2)#6}{, #3(#1)#4 to #5(#2)#6}{ and #3(#1)#4 to #5(#2)#6}
%diagram
\crefformat{diagram}{#2Diagram (#1)#3}
\crefrangeformat{diagram}{#3Diagrams (#1)#4 to #5(#2)#6}
\crefmultiformat{diagram}{#2Diagrams (#1)#3}{ and #2(#1)#3}{, #2(#1)#3}{ and #2(#1)#3}
\crefrangemultiformat{diagram}{#3Diagrams ((#1))#4 to #5((#2))#6}{ and #3(#1)#4 to #5(#2)#6}{, #3(#1)#4 to #5(#2)#6}{ and #3(#1)#4 to #5(#2)#6}



%\usepackage[sortcites=true]{biblatex} % biblatex DOEST WORK WITH LIVE TYPESETTER
\usepackage[nocompress]{cite}
%\bibliographystyle{ieeetr} % trash style mess up the order in bib

\graphicspath{{figures/}}


%%% Todos
\newcommand{\todo}[1]{\vspace{5 mm}\par \noindent
\marginpar{\textsc{\tiny \hspace{0.5cm} ToDo}} \framebox{\begin{minipage}[c]{0.95
\textwidth} \small \tt #1 \end{minipage}}\vspace{5 mm}\par}



\title{Solving PDE}
\author{Rauan}

\begin{document}
\maketitle

\begin{abstract}
\end{abstract}

\tableofcontents

\newpage 

\section{Problem statement}\label{sec:statement}

Let $$\boldsymbol{v}(x,y,t)=\left(u\left(x,y,t\right),v\left(x,y,t\right)\right)$$
 and onsider non-dimensionalized incompressible Navier-Stokes system of equations

\begin{subequations}
\label[pluralequation]{eqs:NSE-dsm-bl}
\begin{align}
\label{eqn:momentum}
\text{Momentum: }	&\frac{\partial \boldsymbol{v}}{\partial t} + \boldsymbol{v} \cdot \nabla \boldsymbol{v} = -\nabla p + \epsilon \nabla \cdot \nabla \boldsymbol{v}, \\ 
					&\epsilon = \frac{1}{\operatorname{Re}},0\leq x \leq 1, 0\leq y \leq 1, t\geq0.\notag\\
\label{eqn:continuity-dsm-bl}
\text{Continuity: }	& \nabla \cdot \boldsymbol{v} = 0, \\ 
					&0\leq x \leq 1, 0\leq y \leq 1, t\geq0. \notag\\
\label[bc]{eqn:NSE-dsm-bl-bc-freestream}
\text{Inlet: } 	& \boldsymbol{v}(t,0,y)=\boldsymbol{v}(t,x,1)=\left(1 + A\cos\left( kx -\omega t + \phi_0   \right),v\right),\\
									& \{A\leq 1,k,\omega, \phi_0\}\subset \mathbb{R}.\notag\\
\label[bc]{eqn:NSE-dsm-bl-bc-noslip}
\text{No-slip wall BC: } & \boldsymbol{v}(t,x,0)=(0,0). \\
\label[bc]{eqn:NSE-dsm-bl-bc-open}
\text{Outlet and freestream BC: } 	&\text{Artificial boundary conditions.} \\
					&\text{}\notag\\
\label{eqn:IC}
\text{Initial condition: } &\boldsymbol{v}(0,x,y)=(1,0).
\end{align}
\end{subequations}

We are to determine appropriate boundary conditions for top side of the domain. 
Taking derivative of \cref{eqn:NSE-dsm-bl-bc-freestream} w.r.t $x$
\begin{equation}
	\frac{\partial u}{\partial x} = -kA\sin\left( kx -\omega t + \phi_0  \right),
\end{equation}
and making use of continuity~\cref{eqn:continuity-dsm-bl} leads to
\begin{equation}\label{eqn:dv-dy}
	\frac{\partial v}{\partial y} = kA\sin\left( kx -\omega t + \phi_0  \right).
\end{equation}
We can integrate \cref{eqn:dv-dy} over $d y$ to obtain general expression for vertical velocity component 
\begin{equation}\label{eqn:v-top}
	v(x,y,t)= ykA\sin\left( kx -\omega t + \phi_0   \right)+g(x,t).
\end{equation}

In case of $g(x,t)=0$, plugging \cref{eqn:v-top} together with \cref{eqn:NSE-dsm-bl-bc-freestream} into \cref{eqn:momentum} leads to 

\begin{equation}\label{eqn:x-mom-plugged}
	\frac{\partial p}{\partial x} = 0
\end{equation}
for $x$-momentum. If we now take $\frac{\partial }{\partial y}$ of \cref{eqn:x-mom-plugged} and plug into $\frac{\partial}{\partial x}$ of $y$-momentum we will obtain  

\begin{equation}
\begin{aligned}
	\frac{A\,k^4\,y\,\cos\left(k x+\omega t + \phi_0 \right)}{\mathrm{Re}}-A\,k^2\,\omega\,y\,\sin\left(k x+\omega t + \phi_0 \right)&
	\\-A\,k^3\,y\,\sin\left(k x+\omega t + \phi_0 \right)\,\left(A\,\cos\left(k x+\omega t + \phi_0 \right)+1\right)&
	\\+A^2\,k^3\,y\,\cos\left(k x+\omega t + \phi_0 \right)\,\sin\left(k x+\omega t + \phi_0 \right)&=0,
\end{aligned}
\end{equation}
which is obviously not always true for all $A,k,\omega$. Therefore $g(x,t)\neq0 \quad \forall x,t$.

Keeping $g(x,t)$ term and repeating the above process leads to third order PDE
\begin{equation}\begin{aligned}\label{eqn:g-third}
	\left(\frac{\partial ^2}{\partial x^2} g\left(x,t\right)-A\,k^3\,y\,\sin\left(k x+\omega t + \phi_0 \right)\right)\,\left(A\,\cos\left(k x+\omega t + \phi_0 \right)+1\right)&
	\\-\frac{1}{\mathrm{Re}}\left( \frac{\partial ^3}{\partial x^3} g\left(x,t\right)-A\,k^4\,y\,\cos\left(k x+\omega t + \phi_0 \right)\right)&
	\\+\frac{\partial }{\partial x} \frac{\partial }{\partial t} g\left(x,t\right)+A\,k^2\,\cos\left(k x+\omega t + \phi_0 \right)\,\left(g\left(x,t\right)+A\,k\,y\,\sin\left(k x+\omega t + \phi_0 \right)\right)&
	\\-A\,k^2\,w\,y\,\sin\left(k x+\omega t + \phi_0 \right)&=0.
\end{aligned}
\end{equation}

We now need to determine boundary conditions for \cref{eqn:g-third}. It is known that at the left boundary $v(0,y,t)=0$. Then \cref{eqn:v-top}  at y=1 gives 
\begin{equation}
	g(x=0,t)=-kA\sin\left(-\omega t + \phi_0   \right).
\end{equation}
Outlet \cref{eqn:NSE-dsm-bl-bc-open} $\frac{\partial v}{\partial x}=0$ at $x=1,y=1$ with \cref{eqn:v-top} leads to 
\begin{equation}
	\frac{\partial }{\partial x}g(x=1,t)=-k^2A\cos\left( k -\omega t + \phi_0   \right).
\end{equation}
Initial condition at $t=0$ we assumed (\cref{eqn:IC}) to be $v(x,y=1,t=0)=0$, then \cref{eqn:v-top} provides us
\begin{equation}
	g(x,t=0)=-kA\sin\left( kx + \phi_0   \right).
\end{equation}


It seems to be possible to obtain pressure values up to a constant if we take $u(x,y=1,t)=1+A\cos(kx+\omega t), v(x,y=1,t)=-Ak\sin(kx+\omega t)$ and plug into $x$-momentum \cref{eqn:momentum}. Namely,
\begin{equation}
	\frac{\partial u}{\partial t}+u\frac{\partial u}{\partial x}+v\frac{\partial v}{\partial y}=-\frac{\partial p}{\partial x}+\epsilon \left( \frac{\partial ^2 u}{\partial x^2}+\frac{\partial ^ 2 u}{\partial y^2} \right)
\end{equation}
becomes
\begin{equation}
	\frac{\partial p}{\partial x}=-A\,w\,\sin\left(k\,x+t\,w\right)+\epsilon{A\,k^2\,\cos\left(k\,x+t\,w\right)}-A\,k\,\sin\left(k\,x+t\,w\right)\,\left(A\,\cos\left(k\,x+t\,w\right)+1\right),
\end{equation}
which we can integrate over $dx$ and to get expression for $p(x,y,t)$ up to a constant across the top boundary.

\end{document}
